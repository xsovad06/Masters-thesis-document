\kap[chapter:conclusion]{Conclusion}
The development of a {\bf Unified Format} for {\bf human pose estimation dataset creation} represents a significant advancement in {\bf simplifying and standardizing} the {\bf aggregation of results} from diverse \NN\ models. By addressing the variations in output formats among existing models like MoveNet, PoseNet, and MMPose, the Unified Format harmonizes these outputs into a cohesive structure. This achievement streamlines the dataset generation process, enabling the training of tailored models for specific detection tasks.

The evaluation of the Unified Format's performance revealed promising results, particularly in metrics like {\bf \APE} ({\bf 3.3\%}) and {\bf \MSE} ({\bf 1213.84}), where low values indicate close alignment between predicted and ground truth keypoints. The {\bf \OKS} metric ({\bf 0.23}), though useful for measuring similarity, faced challenges because of strict evaluation standards and differences in predictions between models. Nonetheless, the Unified Format demonstrated {\bf satisfactory performance} in accurately {\bf localizing keypoints}, especially when compared to individual model outputs.

Despite the overall success of the Unified Format, several areas warrant further attention for refinement. Firstly, {\bf enhancing} the {\bf unification process} to better handle instances where individual models produce {\bf divergent results} could improve overall accuracy. This could involve refining the weighting scheme for averaging predictions or implementing {\bf adaptive strategies} to accommodate varying model outputs.

Additionally, investigating techniques to mitigate errors introduced during the unification process, such as erroneous pose predictions, could lead to more robust dataset generation. Strategies like {\bf outlier detection} or {\bf dynamic thresholding} based on model confidence scores may help improve the quality of Unified Format outputs.

Furthermore, exploring {\bf alternative evaluation metrics} or refining existing ones, particularly {\bf \OKS}, to better reflect the nuances of pose estimation accuracy could enhance the assessment process. This may involve adjusting keypoint similarity criteria or incorporating contextual information to account for pose variations in real-world scenarios.

In summary, the practical part of the thesis has successfully addressed the challenges associated with human pose estimation dataset creation by proposing and implementing a {\bf Unified Format}. This format facilitates the aggregation of outputs from multiple \NN\ models, streamlining the dataset generation process. Evaluation results indicate promising performance, with low \APE\ and \MSE\ values demonstrating close alignment between predicted and ground truth keypoints. While challenges remain, continued refinement and exploration of techniques offer opportunities to further improve the Unified Format and advance the field of human pose estimation.
