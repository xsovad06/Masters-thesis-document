\kap{Introduction}

\pkap[section:motivation]{Motivation and Basic Objectives of the Work}

The field of {\bf computer vision} has witnessed rapid evolution, serving as the foundation for understanding visual information in images and videos (\scc Szeliski, 2010). Within this context, the accurate detection of the {\bf human skeleton} holds immense potential for applications ranging from autonomous systems to healthcare. The motivation driving this master's thesis is to provide an efficient method for {\bf creating specialized datasets} for {\bf human skeleton detection} under realistic conditions through the application of existing {\bf neural networks} (\NN\-s).

The evolution of {\bf Human Pose Estimation} has revolutionized its applicability in real-world scenarios such as smart surveillance, public safety, and medical assistance. However, despite achieving impressive accuracy rates on popular datasets, the translation of these results to real-world settings remains a challenge due to the scarcity of high-quality datasets with {\bf human pose annotations}. This scarcity arises from the costly and time-consuming nature of dataset creation, resulting in real-world applications often being trained on datasets that may not adequately represent the deployment environment (\scc Alinezhad Noghre et al., 2022).

The disparity between training data and real-world inference data frequently leads to high-accuracy models failing to perform as expected in practical applications, particularly in scenarios involving {\bf crowded scenes}, {\bf heavily occluded individuals}, or subjects positioned at a significant {\bf distance from the camera}. Existing datasets attempt to address specific challenges individually, but their varying skeletal structures and limited scope make them inadequate for training a unified model (\scc Alinezhad Noghre et al., 2022).

Training a \NN\ for {\bf human skeleton detection} is inherently challenging. It necessitates the availability of hardware capable of capturing the human body's spatial position through sensors placed on key body points, which are crucial for the detection process. Acquiring sensor data is essential for constructing a comprehensive dataset for the \NN\ training. However, the generation of training data often occurs in controlled "laboratory conditions," using props and actors (\scc Yang, 2018). Consequently, the creation of such a model becomes resource-intensive, requiring significant investments in time, computational resources, human effort, and hardware. Furthermore, the model's accuracy is constrained by the level of correlation between simulated activities and real-world conditions. Another method for creating specialized datasets is manual annotation. However, this process is resource-intensive and time-consuming.

Existing models for {\bf human skeleton detection} exhibit limited accuracy for specific use cases due to training in artificially created conditions (\scc Toshev et al., 2014). The proposed approach involves leveraging existing \NN\ models and combining their functionalities without intervention or retraining. To construct a training dataset, real-world data, such as videos capturing falls in nursing homes, can be used. Existing \NN\ models will extract information about the skeleton from these data, which can then be utilized to train a new model to enhance accuracy in the production environment.

This thesis introduces a cost-effective approach to generating datasets tailored for real-world applications. This method offers flexibility, allowing any detection model to be utilized according to the specific requirements of the dataset. For instance, in creating a dataset for detecting individuals in shopping malls, where subjects may be distant from the camera, a combination of models designed for crowded scenes and long-distance detection would be employed.

\pkap[section:current-state]{Current State and Problem to Be Addressed}
This thesis aims to address the current limitations in accuracy and practicality associated with human skeleton detection training methodologies. A notable gap exists in the market for tools and methodologies specifically designed to train models for this task using pre-existing\NN\-s (\scc Yang et al., 2016). While existing tools focus on model optimization, compression, and transfer learning, there's a lack of knowledge regarding approaches that combine these \NN\-s for specialized dataset generation.

To bridge this gap, the following approach will be undertaken:

\startitemize[n]
    \item {\bf Analysis:} Existing training methodologies for human skeleton detection will be analyzed to identify limitations in accuracy and practicality.
    \item {\bf Proposal:} A novel approach will be proposed that leverages the strengths of existing \NN\-s to generate high-quality annotations specifically tailored for human skeleton detection datasets.
    \item {\bf Implementation:} The proposed approach will be developed and implemented as a tool or framework.
    \item {\bf Testing:} The effectiveness of the proposed approach will be rigorously evaluated by calculation of suitable metrics which will describe the performance of the implemented approach to dataset generation.
\stopitemize
