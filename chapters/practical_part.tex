\kap[chapter:practical-part]{Practical Part}
This chapter comprehensively examines the various stages involved in creating a custom human pose estimation dataset. The initial phase leverages existing models outlined in \in{Section}[section:existing-nns] (on \at{page}[section:existing-nns]). Subsequently, these models are integrated into a unified tool for custom dataset creation. Additionally, the tool's performance is evaluated using two metrics, \APE\ and \OKS\.

Throughout this thesis, several implementation challenges emerged, leading to certain technical limitations outlined in a dedicated \in{Section}[section:problems-limitations] on \at{page}[section:problems-limitations].

% ------- Section ------- %
\pkap[section:proposal]{Solution Proposal}
This section will suggest possible solutions for the specific dataset generation using existing \NN\ detection models.

The preceding chapter offered an analysis of the current market landscape, exploring available technologies and datasets, along with their inherent limitations. As highlighted in \in{Section}[section:pose-estimation-challenges], existing datasets lack the complexity and robustness necessary for training a comprehensive detection model capable of excelling in production environments.

To address this shortfall, the concept of a cost-efficient method for generating a tailored dataset using open-source tools and models was explored. A significant advantage of this approach lies in its flexibility; it is not restricted to specific models and can be fully customized to suit specific requirements.

The production data is subject to the execution of existing models designed to detect human instances and estimate their poses. Currently, multiple detection formats are available. It is imperative to propose a Unified Format for these detections, considering their respective keypoint structures and identifying a common skeleton. Once the Unified Format is established, the process of unification can commence. This entails utilizing the existing detections from the initial step.

Sequential processing of these detections is essential, wherein corresponding frames are loaded from each detection model. Subsequently, individual human instances must be matched. This can be accomplished through comparison of instance bounding boxes or individual keypoint analysis. Once instances are paired, aggregation occurs. Aggregation methods may vary, ranging from weighted to simple averaging, contingent upon the utilized models and their respective accuracy levels.

Through this process, the entire dataset undergoes processing, leading to the attainment of aggregated detection results. By processing one source frame at a time, all detections from individual models are transformed into the Unified Format. At this juncture, evaluation becomes necessary. The Unified Format must be compared against an annotated dataset, providing ground truth data for assessment purposes.

% ------- Section ------- %
\pkap[section:solution-implementation]{Solution Implementation}
\obrazek{overall-process}{Graph of the process introducing individual stages of this thesis}{figures/overall-process.png}{width=38cc}

The moment has arrived to delve into the realm of implementation. Within this section, an overview of the entire process will be presented, aiming to enhance comprehension of the thesis as a cohesive entity. Central to this elucidation will be the graphical representation of the sequential steps undertaken in this thesis. Each of these steps will be comprehensively expounded upon in subsequent sections, offering detailed insights into their execution.

To elucidate the process further, refer to \in{Figure}[overall-process]. The initial phase entails employing the existing skeleton detection models outlined in Sections \in[section:movenet], \in[section:posenet], and \in[section:mmpose]. This step involves the utilization of custom-implemented scripts designed to execute detections using the aforementioned models, as elaborated in the subsequent \in{Section}[section:individual-models-detection]. The detection process is conducted on the evaluation subset comprising 5,000 images from the COCO dataset. Outputs from individual detections are stored in distinct JSON files, the format of which is detailed in \in{Code}[code:detection-script-output].

These JSON files serve as inputs for the Unification script, which also requires keypoint map files for each detection model. These map files facilitate the transformation of native model outputs into the unified format mandated by the script. The Unification script aggregates individual detections into a specified format, generating a JSON file for each image (or video, depending on the script argument). Optionally, these unified format JSON files can be utilized for detection rendering, a functionality provided by the custom Render script.

Concurrently, the COCO WholeBody annotations (comprising 133 keypoints) undergo processing in a custom script. This script takes a single JSON file containing annotations for the entire subset (either training or validation) and produces individual JSON outputs for each image within the subset. Additionally, it generates detections in the same format as the Unification script, consolidating annotations for various body parts into a simplified 27 keypoint format. Optionally, these annotations can be rendered onto the subset images for visual analysis.

The final step entails evaluating the performance of the unified format. For this purpose, a custom script is employed, which compares the annotations (ground truth) with the unified format detections (predictions), calculating various statistical metrics detailed in \in{Section}[section:evaluation].

% ------- Section ------- %
\pkap[section:individual-models-detection]{Individual Models Detection}
This section introduces an approach to implementing individual model detection. It details key concepts and tools for creating complex detection scripts in the {\bf Python} programming language.

The primary tool for implementation is Python and its extensive libraries. The following libraries were used in the initial step of creating detection scripts for individual models:

\startitemize[n]
    \item PoseNet:
        \startitemize[1]
            \item {\bf {\em Mediapipe}} - A~library providing the PoseNet model and tools for drawing detections on images.
        \stopitemize
    \item MoveNet:
        \startitemize[1]
            \item {\bf {\em Tensorflow}} - A~library used for working with image data structures.
            \item {\bf {\em Tensorflow Hub}} - A~repository of pre-trained machine learning models containing the MoveNet model.
        \stopitemize
    \item MMPose:
        \startitemize[1]
            \item {\bf {\em MMCV}} - A~foundational library for computer vision research.
            \item {\bf {\em MMPOSE}} - An open-source toolbox for pose estimation based on {\bf PyTorch}.
            \item {\bf {\em MMDET}} - An open-source object detection toolbox based on {\bf PyTorch}.
        \stopitemize
\stopitemize

Several other libraries are employed in the scripts to furnish essential functionality. Here is a list of the most crucial libraries: {\bf OS}, {\bf Sys}, {\bf OpenCV} and {\bf Numpy}. Additionally, the {\bf typing} library serves as a helper, primarily aimed at providing type hints and enhancing clarity regarding input/output structures. For this purpose, custom datatypes were devised, namely:

\sourcecode[][code:data-types]{Custom Datatypes}{}{src:data-types}{}

These custom datatypes address the detection output format and are organized hierarchically. The first, {\bf DetectionInstance}, represents a single human instance in the image, containing a list of keypoints, as described in \in{Subsection}[subsection:individual-models-detection]. Then follows the {\bf Frame}, which is a list of {\em detection instances}. At the top of this hierarchy lies the {\bf Detection} itself, which is a dictionary of the {\em frames}. Lastly, the final custom datatype, {\bf SkeletonEdgesPainting}, is a structure devised to store skeleton colour codes, assigning a single colour to each keypoint pair.

All these custom datatypes are utilized across various scripts to support type hints and enhance clarity regarding the utilized data structures. Subsequent references will be made to these types.

The main idea behind all detection scripts is to provide a very similar interface with the same functionality. Each detection model uses different resources, meaning the detection part itself is different for every case and needs to be implemented separately. Similarly, the model initialization parts are different. However, the script skeleton remains the same for every model.

The scripts utilize common custom functions gathered in the {\em utils} package. These methods manipulate images to provide the same frame format for every model. Additionally, they include helper functions for processing script arguments, such as handling processed file paths, output file names, and exporting detections to JSON files. Notably, the functions for drawing keypoints and skeletons on detected human instances are specific to the MoveNet model, as there is no library with this functionality for MoveNet, unlike the other two models.

See the script structure in the following code:

\sourcecode[][code:detection-script-backbone]{Simplified Detection Script Backbone (Pseudocode)}{}{src:detection-script-backbone}{}

The detection scripts process various file types based on the provided program arguments. They accept both image and video files. If the input is a directory, the script iterates through all video and image files within it, executing detection on each. Refer to the single-frame detection function \in{Code}[code:detection-script-frame] for a deeper understanding of the core detection implementation.

\sourcecode[][code:detection-script-frame]{Python Implementation of MoveNet Detection Function.}{}{src:detection-script-frame}{}

This function demonstrates the concept behind image detection execution for the MoveNet model. Similar approaches are used for MMPose and PoseNet models, with variations in the specific detection implementation and visualization. The first task is to load the frame or image. Optionally, a function for resizing the frame is available if it's too large. Then follows the pose detection itself. Program arguments manage actions like saving the detection file (JSON), the frame with drawn detections, and displaying the drawn detection frame. Optionally, correctly formatted detections are exported to a JSON file. The functions for drawing the detections onto the frame are then executed, followed by optional saving or displaying of the frame.

% ------- Sub-section ------- %
\ppkap[subsection:individual-models-detection]{Detection Format}
This subsection will take a close look into the individual detection models output format which is crucial for further processing and creation of unified format described in the \in{Section}[section:unified-format] on \at{page}[section:unified-format].

The detection scripts described in the previous section produce the same detection output format for every detection model. To understand the detection JSON file format, see the following example:

\sourcecode[][code:detection-script-output]{Detection Script Output Format (Example)}{}{src:detection-script-output}{}

This format pertains to the {\em single-frame} detection mode. When conducting detection on a video file without the {\em ---single--frame--output} option, the entire detection process is encapsulated within an additional dictionary to distinguish individual video frames. However, for this thesis, which focuses on dataset creation, there is no necessity to save the frame sequence. The {\em---single--frame--output} argument was implemented to segment the video file into individual frames and generate dedicated images for easily processed detection files.

The top level of the output format example includes a {\bf list} of {\bf detection instances} (objects). Each detection instance consists of the keypoints {\bf list}. The number of keypoints varies depending on the chosen detection model. A~single {\bf keypoint} is represented by a {\bf list} of three {\bf float} values. The first two values represent the {\bf coordinates} in pixels, and the last value signifies the {\bf visibility}. Additionally, the MMPose model produces the \BBOX\ position for the instance which is not present in the other model's output.

% ------- Section ------- %
\pkap[section:unified-format]{Created Unified Format}
A~highly cost-effective method of obtaining a custom dataset with real-life footage data is to leverage existing \NN\-s to generate labels. This approach enables the training of a new model specifically tailored to the target detection task. However, for effective training, a unified format is required to aggregate the results of these individual models. This section precisely addresses the concept of a unified format introduced in this thesis, capitalizing on the strengths of existing models. As explained in the previous chapter, each model estimates a different number of keypoints, emphasizing different qualities. Refer to \in{Table}[tab:format-comparison] for a comprehensive understanding of these differences.

The rationale behind the unified format is to identify commonalities among individual formats and address their variations. Essentially, the common format is based on {\bf MoveNet}, which comprises {\bf 17} keypoints, excluding {\em hands} and {\em feet} estimation compared to the unified format. {\bf PoseNet} introduces additional eye keypoints (compared to unified format) that need elimination, while {\bf MMPose} includes {\bf 107} unnecessary keypoints, particularly detailed facial keypoints and non-crucial hand keypoints (individual joints of each finger). The unified format optimally encompasses {\bf 27} keypoints, providing satisfactory detail for hands, feet, and face. Refer to \in{Figure}[unified-format-structure] for a visual representation of the structure. You can see that the format is straightforward. It offers basic pose representation with sufficient detail to both hands and feet.

Another crucial aspect is the accurate aggregation of individual model estimations, encompassing both the coordinates of keypoints and their visibility values. To address this, we introduce a weighted average, mitigating weaknesses in faster models such as MoveNet and PoseNet. Given that these models are designed for real-time estimation, and we utilize the "lightning" model version for PoseNet, accuracy is inherently limited. Detailed values for the weighted average are available in \in{Table}[tab:format-comparison]. The assignment of the highest weight to the {\bf MMPose} model is justified by its superior accuracy with \APE\ metric described in the \in{Section}[section:evaluation]. This approach ensures the uniformity of estimations made by individual models across the entire dataset, as prepared in the previous section.

\TABULKA[][tab:format-comparison]{Comparison of the individual models detection format}
    \setupTABLE[r][1][style=bold]
    \setupTABLE[c][each][offset=3dd]
    \setupTABLE[frame=off]
    \setupTABLE[r][1][topframe=on,bottomframe=on]
    \setupTABLE[c][each][leftframe=on]
    \setupTABLE[c][1][leftframe=off]
    \setupTABLE[c][2,3,4][align=middle]
    \bTR\bTD Model\eTD\bTD              Keypoints\eTD\bTD  Multi-person detection\eTD\bTD  Weight\eTD\eTR
    \bTR\bTD PoseNet\eTD\bTD            33\eTD\bTD         No\eTD\bTD                      0.3\eTD\eTR
    \bTR\bTD MoveNet\eTD\bTD            17\eTD\bTD         No\eTD\bTD                      0.2\eTD\eTR
    \bTR\bTD MMPose\eTD\bTD             133\eTD\bTD        Yes\eTD\bTD                     0.5\eTD\eTR
    \bTR\bTD Unified Format\eTD\bTD     27\eTD\bTD         Yes\eTD\bTD                     --\eTD\eTR

\obrazek{unified-format-structure}{Unified format structure with IDs to each keypoint}{figures/unified-detection-structure.png}{width=\makeupwidth}

The weighted average is not applied in scenarios involving the processing of keypoints on the hands or feet. This limitation arises from the fact that the {\bf MoveNet} model does not provide these keypoints, necessitating a basic average calculation of the two values. In other words, the weighted average is applied only when there is data from all three models.

To clarify and explain the detection outcomes, refer to Figures \in[compose-fig:coco-detection], \in[compose-fig:mmpose-detection] and \in[compose-fig:unified-detection]. The initial depiction presents the original COCO WholeBody annotations in a simplified format, mirroring the Unified Format with 27 keypoints. All individuals within the image are annotated, meticulously considering the overlap and occlusion of human figures. Additionally, non-visible skeleton edges are omitted. Notably, no facial keypoints are detected, aligning with the absence of visible faces, thus demonstrating accurate annotation.

Subsequently, \in{Figure}[compose-fig:mmpose-detection] portrays a detection outcome from the MMPose model. Regrettably, the individual performing a deep squat remains undetected, while facial keypoints are erroneously identified for two adjacent individuals facing away from the camera. Despite these anomalies, the overall detection performance is commendable.

\obrazkyvedlesebe[3][top]{compose-fig:coco-detection;compose-fig:mmpose-detection;compose-fig:unified-detection}
    {
        COCO WholeBody annotation simplified to unified format;
        MMPose model detection with 133 keypoints format;
        Unified format detection with 27 keypoints;
    }
    {
        figures/detection-comparison/000000002153-coco-annotation.jpg;
        figures/detection-comparison/000000002153-mmpose-prediction.jpg;
        figures/detection-comparison/000000002153-unified-prediction.jpg;
    }
    {
        height=9.5cc;
        height=9.5cc;
        height=9.5cc;
    }

Lastly, the Unified Format, primarily informed by the MMPose model, is depicted in the third figure. Notably, a single individual in the upper portion of the image remains undetected. This occurrence is attributed to the unification script, which employs an {\bf average instance visibility threshold}, set experimentally at {\bf 40\%}, to determine inclusion in the unified format. This threshold serves as a filter for wrongly detected human instances by one of the models. Usually this occured by the use of MMPose model which is multi-person detector and this is Nevertheless, the precision of individual detections is highly satisfactory, ensuring precise localization of keypoints when individuals are successfully detected within the image. This is proven in the evaluation \in{Section}[section:evaluation] at \at{page}[section:evaluation].

% ------- Sub-section ------- %
\ppkap[subsection:unified-format-implementation]{Implementation of the Unified Format}
This subsection aims to provide description for the unification script key concepts alongside the code examples.

% ------- Section ------- %
\pkap[section:evaluation]{Evaluation}
In this section, the COCO WholeBody dataset (Xu et al. 2022) will be used as a corpus for the unified detection format evaluation. Additionally, the manipulations and details regarding the processing and use of the dataset will be investigated.

As for the unified detection format evaluation, the {\bf COCO WholeBody} dataset was used. Specifically the {\bf validation 2017} subset containing {\bf 5000} images. According to a COCO validation annotations only {\bf 2693} images has present humans. The total {\bf APE} was equal to {\bf 4.37\%}. This metric was calculated only on the matching instances of the predicted annotations concerning the ground truth annotations. This means that the differences in the predictions and the ground truth were not taken into account. Specifically the missing or additional (detection) instances in the predictions. But as for the corresponding instances, the evaluation serves as a great indicator of how well the individual detection instances were obtained. The criterion used for defining whether the instances from predictions correspond to the ground truth annotations is the bounding box overlap. The threshold used for the overlap value, {\bf Intersection over Union} (\IoU\), is the {\bf 65\%}. To better understand the \IoU\ metric see the \in{Figure}[iou]. The result is very satisfactory even though the averaging of the keypoints position in the unification process causes some errors.

\obrazek{iou}{Calculation of the Intersection over Union}{figures/iou.png}{width=\makeupwidth}

There was also an evaluation on the \APE\ metric just for the {\bf MMPose} model because only this model provides us with the \BBOX\--es in the detection. The total \APE\ was: {\bf 1.29\%} which provided us with the information about the model precision on individual keypoints of detected instances. This information will be used for setting up the weights for weighted average in the process of unified format creation in the following \in{section}[section:unified-format] on \at{page}[section:unified-format]. Generally, the MMPose model is showing great performance on the detection tasks. Visually, there are only occasional differences in the prediction and the ground truth.

Because of the fact, that human pose estimation is a very complex task, it is required to use multiple metrics to express the performance of the detection models. For this reason, additionally, the \MSE\ and \OKS\ are evaluated as well.

\TABULKA[][tab:evaluation-unified]{Evaulation Statistics for the Unified and MMpose Format}
    \setupTABLE[r][1][style=bold]
    \setupTABLE[c][each][offset=3dd]
    \setupTABLE[frame=off]
    \setupTABLE[r][1][topframe=on,bottomframe=on]
    \setupTABLE[c][each][leftframe=on]
    \setupTABLE[c][1][leftframe=off]
    \setupTABLE[c][2,3][align=middle]
    \bTR\bTD Statistic                                   \eTD\bTD    Unified F.\eTD\bTD      MMPose\eTD\eTR
    \bTR\bTD Total number ground truth images:           \eTD\bTD    2693 \eTD\bTD            2693 \eTD\eTR
    \bTR\bTD Total Average Percentage Error (APE):       \eTD\bTD    2.05\% \eTD\bTD          0.54\% \eTD\eTR
    \bTR\bTD Total Mean Squared Error (MSE):             \eTD\bTD    813.37 \eTD\bTD          70.8 \eTD\eTR
    \bTR\bTD Total Object Kepoint Similarity (OSK):      \eTD\bTD    0.2 \eTD\bTD             0.2 \eTD\eTR
    \bTR\bTD Total average of correct keypoint detected: \eTD\bTD    20.30\% \eTD\bTD         20.31\% \eTD\eTR
    \bTR\bTD Total matching number of poses (instances): \eTD\bTD    5833 (53.01\%) \eTD\bTD  5833 (53.01\%) \eTD\eTR
    \bTR\bTD Total predicted number of poses (instances):\eTD\bTD    7746 (70.39\%) \eTD\bTD  7685 (69.84\%) \eTD\eTR
    \bTR\bTD Total truth number of poses (instances):    \eTD\bTD    11004 \eTD\bTD           11004 \eTD\eTR
    \bTR\bTD Model predicted more poses in image (total):\eTD\bTD    589 (5.35\%)\eTD\bTD     589 (5.35\%) \eTD\eTR
    \bTR\bTD Model predicted less poses in image (total):\eTD\bTD    3847 (34.96\%)\eTD\bTD   3908 (35.51\%) \eTD\eTR

The evaluation statistics for the Unified Format can be found in Table \in{Table}[tab:evaluation-unified]. This table presents various values, ranging from the overall counts of processed images with ground truth annotations to differences in predicted pose counts. Easily interpretable is the \APE\ metric, indicating the average error of predicted keypoint positions relative to ground truth annotations per pose. The value of {\bf 2.05\%} for \APE\ signifies that the predicted detection closely aligns with the ground truth annotation. Conversely, the \OKS\ metric yields a somewhat concerning result, with the value of {\bf 0.2} falling towards the lower end of the metrics interval ($ <0,1> $). Generally, higher values of \OKS\ are preferable, indicative of better alignment. The Evaluation script considers a keypoint similar when the OKS for that particular keypoint exceeds the {\bf 0.5} threshold. For a comprehensive understanding of these values, including \MSE\, refer to examples of images and their individual metrics in Figures \in[compose-fig:mmpose-metric] and \in[compose-fig:mmpose-metric]. Additionally, Table \in{Table}[tab:evaluation-metric-explanation] provides individual statistics for these two detections. The first figure depicts a simplified version of the MMPose model output (comprising {\bf 27 keypoints} in the Unified Format), while the second figure presents the original COCO Wholebody annotations also in the simplified format. This comparison aids in better comprehending the available metrics.

\obrazkyvedlesebe[2][top]{compose-fig:mmpose-metric;compose-fig:coco-metric}
    {
        MMPose simplified output;
        COCO Annotations;
    }
    {
        figures/metrics-explanation/000000235784-mmpose.jpg;
        figures/metrics-explanation/000000235784-coco.jpg;
    }
    {
        height=10.5cc;
        height=10.5cc;
    }

\TABULKA[][tab:evaluation-metic-explanation]{Single Image Metrics Evaluation and Comparison}
    \setupTABLE[r][1][style=bold]
    \setupTABLE[c][each][offset=3dd]
    \setupTABLE[frame=off]
    \setupTABLE[r][1][topframe=on,bottomframe=on]
    \setupTABLE[c][each][leftframe=on]
    \setupTABLE[c][1][leftframe=off]
    \setupTABLE[c][2,3][align=middle]
    \bTR\bTD Statistic                                   \eTD\bTD    MMPose\eTD\bTD     Unified F.\eTD\eTR
    \bTR\bTD Image Average Percentage Error (APE):       \eTD\bTD    1.77\% \eTD\bTD    59.27\% \eTD\eTR
    \bTR\bTD Image Mean Squared Error (MSE):             \eTD\bTD    49.73 \eTD\bTD     37496.87 \eTD\eTR
    \bTR\bTD Image Object Kepoint Similarity (OSK):      \eTD\bTD    0.70 \eTD\bTD      0.70 \eTD\eTR
    \bTR\bTD Image average of correct keypoint detected: \eTD\bTD    70.37\% \eTD\bTD   70.37\% \eTD\eTR

% ------- Section ------- %
\pkap[section:problems-limitations]{Implementation Problems and Technical Limitations}
Throughout the course of this thesis, there have been many technical limitations and implementation struggles. This section will list some of them.

The initial plan was to utilize the {\bf BodyPoseNet} (NVIDIA, 2024) detection model from {\bf NVIDIA}, which supports multi-person detection and boasts high power and precision. However, after numerous unsuccessful attempts to implement BodyPoseNet detection script, a decision was made to move forward with a different model to avoid further time constraints. The primary challenge was that BodyPoseNet necessitated an NVIDIA GPU unit in conjunction with the DeepStream toolkit. Unfortunately, the development environment for this thesis consisted of macOS {\bf Sonoma} on a {\bf 16-inch MacBook Pro with M1 Max chip and 64GB of RAM}, which lacks an NVIDIA GPU. The initial attempts to implement BodyPoseNet detection involved a virtual machine running {\em Ubuntu 22.04} on {\em Parallels Desktop}. Subsequently, this same virtual machine environment was used for all other detection models explored. Once it was discovered that detection could be performed directly on the native MacBook environment, the virtual machine was no longer required.

\obrazkyvedlesebe[2*2][top]{compose-fig:mmpose-problem;compose-fig:posenet-problem;compose-fig:unified-problem;compose-fig:coco-problem}
    {
        MMPose;
        PoseNet;
        Unified Format;
        COCO Annotations;
    }
    {
        figures/instance-mapping-problem/000000065350-mmpose.jpg;
        figures/instance-mapping-problem/000000065350-posenet.jpg;
        figures/instance-mapping-problem/000000065350-unified.jpg;
        figures/instance-mapping-problem/000000065350-coco.jpg;
    }
    {
        height=10.5cc;
        height=10.5cc;
        height=10.5cc;
        height=10.5cc;
    }

There are also som implementation limitations regarding the unified format creation. Problem occures when you are averaging the detection together, but some models does not provide you with the \BBOX\ values. For example, both PoseNet and MoveNet does not generate the \BBOX\ which results in difficult mapping of the appropriate detection instances for averaging. There are some examples from the COCO evaluation subset, where this phenomenon is really significant. This issue is visible in the Figures \in[compose-fig:mmpose-problem], \in[compose-fig:posenet-problem] and finally \in[compose-fig:unified-problem]. The mapping of the jumping person from PoseNet model in the middle of the figure is to a wrong instance from the MMPose mode, which is a guy in the right. This results in averaging of those two detection and put the detection in between in the Unified Format output. The unified version of the detection is bit deformed but it is cleary visible, how the averaging process combined the original detections. In this case the MoveNet model does not provided any results so the unification script took into an account only MMPose and PoseNet detections.

This issues could be solved by the implementation of the custom function, which would take two instances from the detection scripts and return a similarity constant. The constant would serve for decision if the two detection instances are corresponding. While the MMPose model provide the \BBOX\ of the instance, this could be used for area of comparison for individual keypoints from the other two models. If a certain percentage of keypoints fit into a \BBOX\ (for example 65\%, similarly to a \IoU\ threshold), the instances are considered to be a corresponding ones.

There are also some implementation limitations regarding the creation of the unified format. A~problem arises when averaging detections together, especially when some models do not provide \BBOX\ values. For instance, both PoseNet and MoveNet do not generate \BBOX\ information, making it challenging to map the appropriate detection instances for averaging. This issue is particularly pronounced in examples from the COCO evaluation subset, as illustrated in Figures \in[compose-fig:mmpose-problem], \in[compose-fig:posenet-problem], and \in[compose-fig:unified-problem].

In Figure \in[compose-fig:posenet-problem], for instance, the mapping of the jumping person from the PoseNet model in the middle of the figure is incorrectly associated with a different instance from the MMPose model, depicted as a person on the right. Consequently, the averaging of these two detections results in the placement of the detection in between in the Unified Format output. Although the unified version of the detection appears slightly distorted, it clearly demonstrates how the averaging process combines the original detections. In this specific case, the MoveNet model did not provide any results, prompting the unification script to consider only MMPose and PoseNet detections.

These issues could be addressed by implementing a custom function that compares two instances from the detection scripts and returns a similarity constant. This constant would serve as a basis for determining if the two detection instances correspond. Utilizing the \BBOX\ information provided by the MMPose model, for example, could facilitate comparison by defining an area of overlap for individual keypoints from the other two models. If a certain percentage of keypoints fit into a \BBOX\ (for example, 65\%, similar to an \IoU\ threshold introduced in \in{Section}[section:evaluation]), the instances would be considered corresponding.
