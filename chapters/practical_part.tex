\kap[chapter:practical-part]{Practical part}
In this chapter, we will comprehensively examine the various stages involved in training a new model for human pose estimation. The initial phase entails the preparation of a custom dataset, followed by the utilization of an existing model outlined in \in{Section}[section:existing-nns] on \at{page}[section:existing-nns]. Subsequently, a unified format for pose estimation is introduced to aggregate results from each model. Finally, the complete new model is trained using the prepared annotated dataset.

\pkap[section:dataset]{Dataset}

\pkap[section:unified-format]{Created Unified Format}
The {\em unified format} structure can be found in \in{Figure}[unified-format-structure].
\obrazek{unified-format-structure}{Unified format structure with IDs to each keypoint}{figures/unified-detection-structure.png}{width=\makeupwidth}

\pkap[section:experiments]{Experiments and Results}

\pkap[section:problems-limitations]{Implementation Problems and Technical Limitations}
